\documentclass{article}

\usepackage{caption}
                
\usepackage[backend=biber,hyperref=false,citestyle=authoryear,bibstyle=authoryear]{biblatex}
                
\bibliography{bibliography}
            
\usepackage{graphicx}
                
\usepackage{calc}
                
\newlength{\imgwidth}
                
\newcommand\scaledgraphics[2]{%
                
\settowidth{\imgwidth}{\includegraphics{#1}}%
                
\setlength{\imgwidth}{\minof{\imgwidth}{#2\textwidth}}%
                
\includegraphics[width=\imgwidth,height=\textheight,keepaspectratio]{#1}%
                
}
            
\begin{document}

\title{Make a Repository}

\maketitle


A repository is the data storage location of your outputted publication.


The reposoitories use Git\footnote{Git is open-source software that both GitHub and GitLab are built on - think of it as a time machine for code and all that could do.} technology which allows for versioning of your publication.


We save to GitHub and GitLab\autocite{PerkelJeffrey2016} - including GitLab.com or another hosted instance of the GitLab Community Edition which is open-source software.

\begin{figure}
\scaledgraphics{d095295c-b3e6-4e52-be14-226bdafee26e.png}{0.5}
\caption*{Figure 1: Octocat}\label{F44428261}
\end{figure}

\begin{figure}
\scaledgraphics{f729a128-33fa-4ec4-92b1-bf0eade111b9.jpg}{0.5}
\caption*{Figure 2: GitLab}\label{F8019541}
\end{figure}


\printbibliography[title={Bibliography}]
\end{document}
