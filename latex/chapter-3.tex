\documentclass{article}

\usepackage{hyperref}
\usepackage{caption}
                
\usepackage[backend=biber,hyperref=false,citestyle=authoryear,bibstyle=authoryear]{biblatex}
                
\bibliography{bibliography}
            
\usepackage{graphicx}
                
\usepackage{calc}
                
\newlength{\imgwidth}
                
\newcommand\scaledgraphics[2]{%
                
\settowidth{\imgwidth}{\includegraphics{#1}}%
                
\setlength{\imgwidth}{\minof{\imgwidth}{#2\textwidth}}%
                
\includegraphics[width=\imgwidth,height=\textheight,keepaspectratio]{#1}%
                
}
            
\begin{document}

\title{Step 1: Make a Repository}

\maketitle


A repository is the data storage location of your outputted publication.


The reposoitories use Git\footnote{Git is open-source software that both GitHub and GitLab are built on – think of it as a time machine for code and all that could do.} technology which allows for versioning of your publication with crytographic IDs.


We save to GitHub and GitLab\autocite{PerkelJeffrey2016} – including \href{GitLab.com}{GitLab.com} or another self-hosted instance of the GitLab Community Edition which is open-source software.

\begin{figure}
\scaledgraphics{5656f6ad-7ef0-4a57-9d81-9fc2ecfc77bc.png}{0.5}
\caption*{Octocat: GitHub's mascot}\label{F44428261}
\end{figure}





\printbibliography[title={Bibliography}]
\end{document}
